\chapter{Cleaning up Python threads}
\label{cleanup}

Compared to a lot of other threading environments, Python threads are
pretty good, but there are a couple of features that annoy me.
Fortunately, you can fix them with a little clean-up code.

\section{Semaphore methods}

First, the methods for Python semaphores are called {\tt acquire}
and {\tt release}, which is a perfectly reasonable choice, but
after working on this book for a couple of years, I am used
to {\tt signal} and {\tt wait}.  Fortunately, I can have it
my way by subclassing the version of Semaphore in the
{\tt threading} module:

\begin{lstlisting}[title={Semaphore name change}]{}
import threading
 
class Semaphore(threading._Semaphore):
    wait = threading._Semaphore.acquire
    signal = threading._Semaphore.release
\end{lstlisting}

Once this class is defined, you can create and manipulate Semaphores
using the syntax in this book.

\begin{lstlisting}[title={Semaphore example}]{}
mutex = Semaphore()
mutex.wait()
mutex.signal()
\end{lstlisting}

\section{Creating threads}

The other feature of the {\tt threading} module that annoys
me is the interface for creating and starting threads.  The
usual way requires keyword arguments and two steps:

\begin{lstlisting}[title={Thread example (standard way)}]{}
import threading

def function(x, y, z):
    print x, y, z

thread = threading.Thread(target=function, args=[1, 2, 3])
thread.start()
\end{lstlisting}

In this example, creating the thread has no immediate effect.
But when you invoke {\tt start}, the new thread executes
the target function with the given arguments.
This is great if you need to do something with the thread
before it starts, but I almost never do.
Also, I think the keyword arguments {\tt target} and {\tt args}
are awkward.

Fortunately, we can solve both of these problems with four
lines of code.

\begin{lstlisting}[title={Cleaned-up Thread class}]{}
class Thread(threading.Thread):
    def __init__(self, t, *args):
        threading.Thread.__init__(self, target=t, args=args)
        self.start()
\end{lstlisting}

Now we can create threads with a nicer interface, and they
start automatically:

\begin{lstlisting}[title={Thread example (my way)}]{}
thread = Thread(function, 1, 2, 3)
\end{lstlisting}

This also lends itself to an idiom I like, which is to create
multiple Threads with a list comprehension:

\begin{lstlisting}[title={Multiple thread example}]{}
threads = [Thread(function, i, i, i) for i in range(10)]
\end{lstlisting}

\section{Handling keyboard interrupts}

One other problem with the {\tt threading} class is that 
{\tt Thread.join} can't be interrupted by Ctrl-C, which
generates the signal {\tt SIGINT}, which Python translates
into a KeyboardInterrupt.

\newpage
So, if you write the following program:

\begin{lstlisting}[title={Unstoppable program}]{}
import threading, time

class Thread(threading.Thread):
    def __init__(self, t, *args):
        threading.Thread.__init__(self, target=t, args=args)
        self.start()

def parent_code():
    child = Thread(child_code, 10)
    child.join()

def child_code(n=10):
    for i in range(n):
        print i
        time.sleep(1)
    
parent_code()
\end{lstlisting}

You will find that it cannot be interrupted with Ctrl-C or
a {\tt SIGINT}\footnote{At the time of this writing, this
bug had been reported and assigned number 1167930, but it was
open and unassigned (\url{https://sourceforge.net/projects/python/}).}.

My workaround for this problem uses {\tt os.fork} and {\tt os.wait},
so it only works on UNIX and Macintosh.  Here's how it works:
before creating new threads, the program invokes {\tt watcher},
which forks a new process.  The new process returns and executes
the rest of the program.  The original process waits for the
child process to complete, hence the name {\tt watcher}:

\begin{lstlisting}[title={The watcher}]{}
import threading, time, os, signal, sys

class Thread(threading.Thread):
    def __init__(self, t, *args):
        threading.Thread.__init__(self, target=t, args=args)
        self.start()

def parent_code():
    child = Thread(child_code, 10)
    child.join()

def child_code(n=10):
    for i in range(n):
        print i
        time.sleep(1)

def watcher():
    child = os.fork()
    if child == 0: return
    try:
        os.wait()
    except KeyboardInterrupt:
        print 'KeyboardInterrupt'
        os.kill(child, signal.SIGKILL)
    sys.exit()

watcher()
parent_code()
\end{lstlisting}

If you run this version of the program, you should be able
to interrupt it with Ctrl-C.  I am not sure, but I think it
is guaranteed that the {\tt SIGINT} is delivered to the
watcher process, so that's one less thing the
parent and child threads have to deal with.

I keep all this code in a file named {\tt threading\_cleanup.py},
which you can download from
\url{greenteapress.com/semaphores/threading\_cleanup.py}


The examples in Chapter~\ref{pysync} are presented with the understanding
that this code executes prior to the example code.
