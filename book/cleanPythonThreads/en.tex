\chapter{Cleaning up Python threads}
\label{cleanup}

Compared to a lot of other threading environments, Python threads are
pretty good, but there are a couple of features that annoy me.
Fortunately, you can fix them with a little clean-up code.

\section{Semaphore methods}

First, the methods for Python semaphores are called {\tt acquire}
and {\tt release}, which is a perfectly reasonable choice, but
after working on this book for a couple of years, I am used
to {\tt signal} and {\tt wait}.  Fortunately, I can have it
my way by subclassing the version of Semaphore in the
{\tt threading} module:

\begin{lstlisting}[title={Semaphore name change}]{}
import threading
 
class Semaphore(threading._Semaphore):
    wait = threading._Semaphore.acquire
    signal = threading._Semaphore.release
\end{lstlisting}

Once this class is defined, you can create and manipulate Semaphores
using the syntax in this book.

\begin{lstlisting}[title={Semaphore example}]{}
mutex = Semaphore()
mutex.wait()
mutex.signal()
\end{lstlisting}

\section{Creating threads}

The other feature of the {\tt threading} module that annoys
me is the interface for creating and starting threads.  The
usual way requires keyword arguments and two steps:

\begin{lstlisting}[title={Thread example (standard way)}]{}
import threading

def function(x, y, z):
    print x, y, z

thread = threading.Thread(target=function, args=[1, 2, 3])
thread.start()
\end{lstlisting}

In this example, creating the thread has no immediate effect.
But when you invoke {\tt start}, the new thread executes
the target function with the given arguments.
This is great if you need to do something with the thread
before it starts, but I almost never do.
Also, I think the keyword arguments {\tt target} and {\tt args}
are awkward.

Fortunately, we can solve both of these problems with four
lines of code.

\begin{lstlisting}[title={Cleaned-up Thread class}]{}
class Thread(threading.Thread):
    def __init__(self, t, *args):
        threading.Thread.__init__(self, target=t, args=args)
        self.start()
\end{lstlisting}

Now we can create threads with a nicer interface, and they
start automatically:

\begin{lstlisting}[title={Thread example (my way)}]{}
thread = Thread(function, 1, 2, 3)
\end{lstlisting}

This also lends itself to an idiom I like, which is to create
multiple Threads with a list comprehension:

\begin{lstlisting}[title={Multiple thread example}]{}
threads = [Thread(function, i, i, i) for i in range(10)]
\end{lstlisting}

\section{Handling keyboard interrupts}

One other problem with the {\tt threading} class is that 
{\tt Thread.join} can't be interrupted by Ctrl-C, which
generates the signal {\tt SIGINT}, which Python translates
into a KeyboardInterrupt.

\newpage
So, if you write the following program:

\begin{lstlisting}[title={Unstoppable program}]{}
import threading, time

class Thread(threading.Thread):
    def __init__(self, t, *args):
        threading.Thread.__init__(self, target=t, args=args)
        self.start()

def parent_code():
    child = Thread(child_code, 10)
    child.join()

def child_code(n=10):
    for i in range(n):
        print i
        time.sleep(1)
    
parent_code()
\end{lstlisting}

You will find that it cannot be interrupted with Ctrl-C or
a {\tt SIGINT}\footnote{At the time of this writing, this
bug had been reported and assigned number 1167930, but it was
open and unassigned (\url{https://sourceforge.net/projects/python/}).}.

My workaround for this problem uses {\tt os.fork} and {\tt os.wait},
so it only works on UNIX and Macintosh.  Here's how it works:
before creating new threads, the program invokes {\tt watcher},
which forks a new process.  The new process returns and executes
the rest of the program.  The original process waits for the
child process to complete, hence the name {\tt watcher}:

\begin{lstlisting}[title={The watcher}]{}
import threading, time, os, signal, sys

class Thread(threading.Thread):
    def __init__(self, t, *args):
        threading.Thread.__init__(self, target=t, args=args)
        self.start()

def parent_code():
    child = Thread(child_code, 10)
    child.join()

def child_code(n=10):
    for i in range(n):
        print i
        time.sleep(1)

def watcher():
    child = os.fork()
    if child == 0: return
    try:
        os.wait()
    except KeyboardInterrupt:
        print 'KeyboardInterrupt'
        os.kill(child, signal.SIGKILL)
    sys.exit()

watcher()
parent_code()
\end{lstlisting}

If you run this version of the program, you should be able
to interrupt it with Ctrl-C.  I am not sure, but I think it
is guaranteed that the {\tt SIGINT} is delivered to the
watcher process, so that's one less thing the
parent and child threads have to deal with.

I keep all this code in a file named {\tt threading\_cleanup.py},
which you can download from
\url{greenteapress.com/semaphores/threading\_cleanup.py}


The examples in Chapter~\ref{pysync} are presented with the understanding
that this code executes prior to the example code.


\chapter{Cleaning up POSIX threads}
\label{ccleanup}

In this section, I present some utility code I use to make
multithreading in C a little more pleasant.  The examples in
Section~\ref{csync} are based on this code.

Probably the most popular threading standard used with C is
POSIX Threads, or Pthreads for short.  The POSIX standard defines
a thread model and an interface for creating and controlling
threads.  Most versions of UNIX provide an implementation of
Pthreads.

\section{Compiling Pthread code}

Using Pthreads is like using most C libraries:

\begin{itemize}

\item You include headers files at the beginning of your
program.

\item You write code that calls functions defined by Pthreads.

\item When you compile the program, you link it with the
Pthread library.

\end{itemize}

For my examples, I include the following headers:

\begin{lstlisting}[title={Headers}]{}
#include <stdio.h>
#include <stdlib.h>
#include <pthread.h>
#include <semaphore.h>
\end{lstlisting}

The first two are standard; the third is for Pthreads and
the fourth is for semaphores.
To compile with the Pthread library in {\tt gcc}, you
can use the {\tt -l}
option on the command line:

\begin{lstlisting}[title={}]{}
gcc -g -O2 -o array array.c -lpthread
\end{lstlisting}

This compiles {\tt array.c} with debugging info and optimization,
links with the Pthread library, and generates an executable
named {\tt array}.

If you are used to a language like Python that provides exception
handling, you will probably be annoyed with languages like C that
require you to check for error conditions explicitly.  I often
mitigate this hassle by wrapping library function calls
together with their error-checking code inside my own functions.
For example, here is a version of {\tt malloc}
that checks the return value.

\begin{lstlisting}[title={}]{}
void *check_malloc(int size)
{
  void *p = malloc (size);
  if (p == NULL) {
    perror ("malloc failed");
    exit (-1);
  }
  return p;
}
\end{lstlisting}


\section{Creating threads}

I've done the same thing with the Pthread functions I'm going to use;
here's my wrapper for {\tt pthread\_create}.

\begin{lstlisting}[title={}]{}
pthread_t make_thread(void *(*entry)(void *), Shared *shared)
{
  int n;
  pthread_t thread;

  n = pthread_create (&thread, NULL, entry, (void *)shared);
  if (n != 0) {
    perror ("pthread_create failed");
    exit (-1);
  }
  return thread;
}
\end{lstlisting}

The return type from {\tt pthread\_create} is {\tt pthread\_t},
which you can think of as a handle for the new thread.  You
shouldn't have to worry about the implementation of {\tt pthread\_t},
but you do have to know that it has the semantics of a primitive
type\footnote{Like an integer, for example, which is what a
{\tt pthread\_t} is in all the implementations I know.}.  That
means that you can think of a thread handle as an immutable
value, so you can copy it or pass it by value without causing
problems.  I point this out now because it is not true for
semaphores, which I will get to in a minute.

If {\tt pthread\_create} succeeds, it returns 0 and my function
returns the handle of the new thread.
If an error occurs, {\tt pthread\_create} 
returns an error code and my function prints an error message
and exits.

The parameters of {\tt pthread\_create} take some
explaining.  Starting with the second,
{\tt Shared}
is a user-defined structure that contains shared variables.
The following {\tt typedef} statement creates the new type:

\begin{lstlisting}[title={}]{}
typedef struct {
  int counter;
} Shared;
\end{lstlisting}

In this case, the only shared variable is {\tt counter}.
{\tt make\_shared} allocates
space for a {\tt Shared} structure and initializes the contents:

\begin{lstlisting}[title={}]{}
Shared *make_shared ()
{
  int i;
  Shared *shared = check_malloc (sizeof (Shared));
  shared->counter = 0;
  return shared;
}
\end{lstlisting}

Now that we have a shared data structure, let's get back to
{\tt pthread\_create}.
The first parameter is a pointer to a function that takes
a {\tt void} pointer and returns a {\tt void} pointer.  If the syntax
for declaring this type makes your eyes bleed, you are not alone.
Anyway, the purpose of this parameter is to specify the function where
the execution of the new thread will begin.  By convention, this
function is named {\tt entry}:

\begin{lstlisting}[title={}]{}
void *entry (void *arg)
{
  Shared *shared = (Shared *) arg;
  child_code (shared);
  pthread_exit (NULL);
}
\end{lstlisting}

The parameter of {\tt entry} has to be declared as a {\tt void}
pointer, but in this program we know that it is really a pointer to a
{\tt Shared} structure, so we can typecast it accordingly and then
pass it along to {\tt child\_code}, which does the real work.

When {\tt child\_code} returns, we invoke {\tt pthread\_exit}
which can be used to pass a value to any thread (usually the
parent) that joins with this thread.  In this case, the child
has nothing to say, so we pass {\tt NULL}.


\section{Joining threads}

When one thread want to wait for another thread to complete,
it invokes {\tt pthread\_join}.
Here is my wrapper for {\tt pthread\_join}:

\begin{lstlisting}[title={}]{}
void join_thread (pthread_t thread)
{
  int ret = pthread_join (thread, NULL);
  if (ret == -1) {
    perror ("pthread_join failed");
    exit (-1);
  }
}
\end{lstlisting}

The parameter is the handle of the thread you want to wait for.
All my function does is call {\tt pthread\_join} and check the
result.


\section{Semaphores}

The POSIX standard specifies an interface for semaphores.
This interface is not part of Pthreads, but most UNIXes
that implement Pthreads also provide semaphores.  If you
find yourself with Pthreads and without semaphores, you
can make your own; see Section~\ref{makeyourown}.

POSIX semaphores have type {\tt sem\_t}.  You shouldn't have
to know about the implementation of this type, but you do
have to know that it has structure semantics, which means that
if you assign it to a variable you are making a copy of the
contents of a structure.  Copying a semaphore is almost certainly
a bad idea.  In POSIX, the behavior of the copy is undefined.

In my programs, I use capital letters to denote types with
structure semantics, and I always manipulate them with pointers.
Fortunately, it is easy to put a wrapper around {\tt sem\_t}
to make it behave like a proper object.  Here is the 
{\tt typedef} and the wrapper that creates and initializes
semaphores:

\begin{lstlisting}[title={}]{}
typedef sem_t Semaphore;

Semaphore *make_semaphore (int n)
{
  Semaphore *sem = check_malloc (sizeof(Semaphore));
  int ret = sem_init(sem, 0, n);
  if (ret == -1) {
    perror ("sem_init failed");
    exit (-1);    
  }
  return sem;
}
\end{lstlisting}

{\tt make\_semaphore} takes the initial value of the semaphore
as a parameter.  It allocates space for a Semaphore, initializes
it, and returns a pointer to {\tt Semaphore}.

{\tt sem\_init} uses old-style UNIX error reporting, which means
that it returns -1 if something went wrong.  Once nice thing
about these wrapper functions is that we don't have to remember
which functions use which reporting style.

With these definitions, we can write C code that almost looks
like a real programming language:

\begin{lstlisting}[title={}]{}
Semaphore *mutex = make_semaphore(1);
sem_wait(mutex);
sem_post(mutex);
\end{lstlisting}

Annoyingly, POSIX semaphores use {\tt post} instead of
{\tt signal}, but we can fix that:

\begin{lstlisting}[title={}]{}
int sem_signal(Semaphore *sem)
{
  return sem_post(sem);
}
\end{lstlisting}

That's enough cleanup for now.

\end{document}

