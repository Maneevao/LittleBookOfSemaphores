\chapter{Semaphores}

In real life a semaphore is a system of signals used to communicate
visually, usually with flags, lights, or some other mechanism.  In
software, a semaphore is a data structure that is useful for solving a
variety of synchronization problems.

Semaphores were invented by Edsger Dijkstra, a famously eccentric
computer scientist.  Some of the details have changed since the
original design, but the basic idea is the same.

\section{Definition}

A semaphore is like an integer, with three differences:

\begin{enumerate}

\item When you create the semaphore, you can initialize its value to
any integer, but after that the only operations you are allowed to
perform are increment (increase by one) and decrement (decrease by
one).  You cannot read the current value of the semaphore.

\item When a thread decrements the semaphore, if the result is
negative, the thread blocks itself and cannot continue until another
thread increments the semaphore.

\item When a thread increments the semaphore, if there are other
threads waiting, one of the waiting threads gets unblocked.

\end{enumerate}

To say that a thread blocks itself (or simply ``blocks'') is to say
that it notifies the scheduler that it cannot proceed.  The scheduler
will prevent the thread from running until an event occurs that causes
the thread to become unblocked.  In the tradition of mixed metaphors
in computer science, unblocking is often called ``waking''.

That's all there is to the definition, but there are some
consequences of the definition you might want to think about.

\begin{itemize}

\item In general, there is no way to know before a thread decrements a
semaphore whether it will block or not (in specific cases you might
be able to prove that it will or will not).

\item After a thread increments a semaphore and another thread gets
woken up, both threads continue running concurrently.  There is no way
to know which thread, if either, will continue immediately.

\item When you signal a semaphore, you don't necessarily know whether
another thread is waiting, so the number of unblocked threads may
be zero or one.

\end{itemize}

Finally, you might want to think about what the value of the
semaphore means.  If the value is positive, then it represents the
number of threads that can decrement without blocking.  If it
is negative, then it represents the number of threads that have
blocked and are waiting.  If the value is zero, it means there
are no threads waiting, but if a thread tries to decrement, it
will block.


\section{Syntax}

In most programming environments, an implementation of semaphores is
available as part of the programming language or the operating system.
Different implementations sometimes offer slightly different
capabilities, and usually require different syntax.

In this book I will use a simple pseudo-language to demonstrate
how semaphores work.  The syntax for creating a new semaphore
and initializing it is
%
\begin{lstlisting}[title={Semaphore initialization syntax}]{}
fred = Semaphore(1)
\end{lstlisting}
%
The function {\tt Semaphore} is a constructor; it creates and
returns a new Semaphore.  The initial value of the semaphore
is passed as a parameter to the constructor.

The semaphore operations go by different names in different environments.
The most common alternatives are 
%
\begin{lstlisting}[title={Semaphore operations}]{}
fred.increment()
fred.decrement()	
\end{lstlisting}
%
and
%
\begin{lstlisting}[title={Semaphore operations}]{}
fred.signal()
fred.wait()	
\end{lstlisting}
%
and
%
\begin{lstlisting}[title={Semaphore operations}]{}
fred.V()
fred.P()	
\end{lstlisting}
%
It may be surprising that there are so many names, but there is a
reason.  {\tt increment} and {\tt decrement}
describe what the operations {\em do}.  {\tt signal} and {\tt wait}
describe what they are often {\em used for}.  And {\tt V} and {\tt P} were
the original names proposed by Dijkstra, who wisely realized that a
meaningless name is better than a misleading name\footnote{If you speak
Dutch, {\tt V} and {\tt P} aren't completely meaningless.}.

I consider the other pairs misleading because {\tt increment} and {\tt
decrement} neglect to mention the possibility of blocking and waking,
and semaphores are often used in ways that have nothing to do with
{\tt signal} and {\tt wait}.

If you insist on meaningful names, then I would suggest these:

\begin{lstlisting}[title={Semaphore operations}]{}
fred.increment_and_wake_a_waiting_process_if_any()
fred.decrement_and_block_if_the_result_is_negative()	
\end{lstlisting}

I don't think the world is likely to embrace either of these names
soon.  In the meantime, I choose (more or less arbitrarily) to use
{\tt signal} and {\tt wait}.


\section{Why semaphores?}

Looking at the definition of semaphores, it is not at all obvious why
they are useful.  It's true that we don't {\em need} semaphores to
solve synchronization problems, but there are some advantages to using
them:

\begin{itemize}

\item Semaphores impose deliberate constraints that help
programmers avoid errors.

\item Solutions using semaphores are often clean and organized,
making it easy to demonstrate their correctness.

\item Semaphores can be implemented efficiently on many systems,
so solutions that use semaphores are portable and usually
efficient.

\end{itemize}

\clearemptydoublepage
