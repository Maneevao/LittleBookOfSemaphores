\chapter{Basic synchronization patterns}

This chapter presents a series of basic synchronization problems and
shows ways of using semaphores to solve them.  These problems include
serialization and mutual exclusion, which we have already seen, along
with others.

\section{Signaling}

Possibly the simplest use for a semaphore is {\bf signaling},
which means that one thread sends a signal to another
thread to indicate that something has happened.

Signaling makes it possible to guarantee
that a section of code in one thread will run before a section of
code in another thread; in other words, it solves the serialization
problem.

Assume that we have a semaphore named {\tt sem} with initial value
0, and that Threads A and B have shared access to it.

\begin{minipage}[t]{2in}
\begin{lstlisting}[title={Thread A}]{}
statement a1
sem.signal()
\end{lstlisting}
\end{minipage}
\hfill
\begin{minipage}[t]{2in}
\begin{lstlisting}[title={Thread B}]{}
sem.wait()
statement b1
\end{lstlisting}
\end{minipage}

The word {\tt statement} represents an arbitrary program statement.
To make the example concrete, imagine that {\tt a1} reads a line
from a file, and {\tt b1} displays the line on the screen.
The semaphore in this program guarantees that Thread A
has completed {\tt a1} before Thread B begins {\tt b1}.

Here's how it works: if thread B gets to the
{\tt wait} statement first, it will find the initial
value, zero, and it will block.  Then when Thread A signals,
Thread B proceeds.

Similarly, if Thread A gets to the signal first then the
value of the semaphore will be incremented, and when Thread
B gets to the wait, it will proceed immediately.
Either way, the order of {\tt a1} and {\tt b1} is guaranteed.

This use of semaphores is the basis of the names {\tt signal}
and {\tt wait}, and in this case the names are conveniently
mnemonic.  Unfortunately, we will see other cases where the
names are less helpful.

Speaking of meaningful names, {\tt sem} isn't one.  When
possible, it is a good idea to give a semaphore a name
that indicates what it represents.  In this case a name like
{\tt a1Done} might be good, so that {\tt a1done.signal()} means
``signal that a1 is done,'' and {\tt a1done.wait()} means
``wait until a1 is done.''


\section{Sync.py}
\label{sync.py}

TODO: write about using sync, starting with signal.py

Why does Thread B signal {\tt initComplete}?



\section{Rendezvous}
\label{rendezvous}

Puzzle: Generalize the signal pattern so that it works both
ways.  Thread A has to wait for Thread B and vice versa.  In other
words, given this code

\begin{minipage}[t]{2in}
\begin{lstlisting}[title={Thread A}]{}
statement a1
statement a2
\end{lstlisting}
\end{minipage}
\hfill
\begin{minipage}[t]{2in}
\begin{lstlisting}[title={Thread B}]{}
statement b1
statement b2
\end{lstlisting}
\end{minipage}
%
we want to guarantee that {\tt a1} happens before {\tt b2} and
{\tt b1} happens before {\tt a2}.  In writing your solution, be sure
to specify the names and initial values of your semaphores
(little hint there).

Your solution should not enforce too many constraints.  For example,
we don't care about the order of {\tt a1} and {\tt b1}.  In your
solution, either order should be possible.

This synchronization problem has a name; it's a
rendezvous.  The idea is that two threads rendezvous
at a point of execution, and neither is allowed to proceed
until both have arrived.


\clearemptydoublepage
\subsection{Rendezvous hint}

The chances are good that you were able to figure out a solution,
but if not, here is a hint.  Create two semaphores, named {\tt aArrived}
and {\tt bArrived}, and initialize them both to zero.

As the names suggest, {\tt aArrived} indicates whether Thread A
has arrived at the rendezvous, and {\tt bArrived} likewise.

\clearemptydoublepage
\subsection{Rendezvous solution}

Here is my solution, based on the previous hint:

\begin{minipage}[t]{2in}
\begin{lstlisting}[title={Thread A}]{}
statement a1
aArrived.signal()
bArrived.wait()
statement a2
\end{lstlisting}
\end{minipage}
\hfill
\begin{minipage}[t]{2in}
\begin{lstlisting}[title={Thread B}]{}
statement b1
bArrived.signal()
aArrived.wait()
statement b2
\end{lstlisting}
\end{minipage}

While working on the previous problem, you might have
tried something like this:

\begin{minipage}[t]{2in}
\begin{lstlisting}[title={Thread A}]{}
statement a1
bArrived.wait()
aArrived.signal()
statement a2
\end{lstlisting}
\end{minipage}
\hfill
\begin{minipage}[t]{2in}
\begin{lstlisting}[title={Thread B}]{}
statement b1
bArrived.signal()
aArrived.wait()
statement b2
\end{lstlisting}
\end{minipage}

This solution also works, although it is probably less
efficient, since it might have to switch between A and B
one time more than necessary.

If A arrives first, it waits for B.  When B arrives, it wakes
A and might proceed immediately to its {\tt wait} in which
case it blocks, allowing A to reach its {\tt signal}, after
which both threads can proceed.

Think about the other possible paths through this code and
convince yourself that in all cases neither thread can
proceed until both have arrived.

\subsection {Deadlock \#1}

Again, while working on the previous problem, you might have
tried something like this:

\begin{minipage}[t]{2in}
\begin{lstlisting}[title={Thread A}]{}
statement a1
bArrived.wait()
aArrived.signal()
statement a2
\end{lstlisting}
\end{minipage}
\hfill
\begin{minipage}[t]{2in}
\begin{lstlisting}[title={Thread B}]{}
statement b1
aArrived.wait()
bArrived.signal()
statement b2
\end{lstlisting}
\end{minipage}

If so, I hope you rejected it quickly, because it has a serious
problem.  Assuming that A arrives first, it will block at its
{\tt wait}.  When B arrives, it will also block, since A wasn't
able to signal {\tt aArrived}.  At this point, neither thread
can proceed, and never will.

This situation is called a {\bf deadlock} and, obviously, it is
not a successful solution of the synchronization problem.  In
this case, the error is obvious, but often the possibility of
deadlock is more subtle.  We will see more examples later.


\section{Mutex}

A second common use for semaphores is to enforce mutual exclusion.
We have already seen one use for mutual exclusion, controlling
concurrent access to shared variables.  The mutex guarantees
that only one thread accesses the shared variable at a time.

A mutex is like a token that passes from one thread to another,
allowing one thread at a time to proceed.  For example, in {\em The
Lord of the Flies} a group of children use a conch as a mutex.  In
order to speak, you have to hold the conch.  As long as only one child
holds the conch, only one can speak\footnote{Although this metaphor
is helpful for now, it can also be misleading, as you will see in
Section~\ref{water}}.

Similarly, in order for a thread to access a shared variable,
it has to ``get'' the mutex; when it is done, it ``releases''
the mutex.  Only one thread can hold the mutex at a time.

Puzzle: Add semaphores to the following example to
enforce mutual exclusion to the shared variable {\tt count}.

\begin{minipage}[t]{2in}
\begin{lstlisting}[title={Thread A}]{}
count = count + 1
\end{lstlisting}
\end{minipage}
\hfill
\begin{minipage}[t]{2in}
\begin{lstlisting}[title={Thread B}]{}
count = count + 1
\end{lstlisting}
\end{minipage}


\clearemptydoublepage
\subsection{Mutual exclusion hint}

Create a semaphore named {\tt mutex} that is initialized
to 1.  A value of one means that a thread may proceed and
access the shared variable; a value of zero means that it
has to wait for another thread to release the mutex.


\clearemptydoublepage
\subsection{Mutual exclusion solution}

Here is a solution:

\begin{minipage}[t]{2in}
\begin{lstlisting}[title={Thread A}]{}
mutex.wait()
    # critical section
    count = count + 1
mutex.signal()
\end{lstlisting}
\end{minipage}
\hfill
\begin{minipage}[t]{2in}
\begin{lstlisting}[title={Thread B}]{}
mutex.wait()
    # critical section
    count = count + 1
mutex.signal()
\end{lstlisting}
\end{minipage}

Since {\tt mutex} is initially 1, whichever thread gets to
the {\tt wait} first will be able to proceed immediately.
Of course, the act of waiting on the semaphore has the effect
of decrementing it, so the second thread to
arrive will have to wait until the first signals.

I have indented the update operation to show that it is contained
within the mutex.

In this example, both threads are running the same code.  This is
sometimes called a {\bf symmetric} solution.  If the threads have to
run different code, the solution is {\bf asymmetric}.  Symmetric
solutions are often easier to generalize.  In this case, the mutex
solution can handle any number of concurrent threads without
modification.  As long as every thread waits before 
performing an update and signals after, then no two threads
will access {\tt count} concurrently.

Often the code that needs to be protected is called the
{\bf critical section}, I suppose because it is critically
important to prevent concurrent access.

In the tradition of computer science and mixed metaphors, there
are several other ways people sometimes talk about mutexes.  In
the metaphor we have been using so far, the mutex is a token
that is passed from one thread to another.

In an alternative
metaphor, we think of the critical section as a room, and
only one thread is allowed to be in the room at a time.
In this metaphor, mutexes are called locks, and a thread
is said to lock the mutex before entering and unlock it while
exiting.  Occasionally, though, people mix the metaphors and
talk about ``getting'' or ``releasing'' a lock, which doesn't
make much sense.

Both metaphors are potentially useful and potentially misleading.
As you work on the next problem, try out both ways of thinking
and see which one leads you to a solution.


\section{Multiplex}

Puzzle: Generalize the previous solution so that it allows multiple
threads to run in the critical section at the same time, but it
enforces an upper limit on the number of concurrent threads.  In other
words, no more than {\tt n} threads can run in the critical section at
the same time.

This pattern is called a {\bf multiplex}.  In real life, the multiplex
problem occurs at busy nightclubs where there is a maximum number of
people allowed in the building at a time, either to maintain fire
safety or to create the illusion of exclusivity.

At such places a bouncer usually enforces the synchronization
constraint by keeping track of the number of people inside
and barring arrivals when the room is at capacity.  Then,
whenever one person leaves another is allowed to enter.

Enforcing this constraint with semaphores may sound difficult, but it
is almost trivial.


\clearemptydoublepage
\subsection{Multiplex solution}

To allow multiple threads to run in the critical section, just
initialize the semaphore to {\tt n}, which is the maximum number
of threads that should be allowed.

At any time, the value of the semaphore represents the
number of additional threads that may enter.  If the value is zero,
then the next thread will block until one of the threads inside
exits and signals.  When all threads have exited the value of the
semaphore is restored to {\tt n}.

Since the solution is symmetric, it's conventional to show only one
copy of the code, but you should imagine multiple copies of the code
running concurrently in multiple threads.

\begin{lstlisting}[title={Multiplex solution}]{}
multiplex.wait()
    critical section 
multiplex.signal()      
\end{lstlisting}

What happens if the critical section is occupied and more than one
thread arrives?  Of course, what we want is for all the arrivals to
wait.  This solution does exactly that.  Each time an arrival joins
the queue, the semaphore is decremented, so that the value of the
semaphore (negated) represents the number of threads in queue.

When a thread leaves, it signals the semaphore, incrementing
its value and allowing one of the waiting threads to proceed.

Thinking again of metaphors, in this case I find it useful
to think of the semaphore as a set of tokens (rather than
a lock).
As each thread invokes {\tt wait}, it picks up one of
the tokens; when it invokes {\tt signal} it releases one.
Only a thread that holds a token can enter the room.  If no
tokens are available when a thread arrives, it waits until
another thread releases one.

In real life, ticket windows sometimes use a system like
this.  They hand out tokens (sometimes poker chips) to
customers in line.  Each token allows the holder to buy a ticket.




\section{Barrier}

Consider again the Rendezvous problem from Section~\ref{rendezvous}.
A limitation of the solution we presented is that it does
not work with more than two threads.

Puzzle: Generalize the rendezvous solution.  Every thread should
run the following code:

\begin{lstlisting}[title={Barrier code}]{}
rendezvous
critical point
\end{lstlisting}

The synchronization requirement is that
no thread executes {\tt critical point} until after all
threads have executed {\tt rendezvous}.

You can assume that there are $n$
threads and that this value is stored in a variable, {\tt n},
that is accessible from all threads.

When the first $n-1$ threads arrive they should block until the $n$th
thread arrives, at which point all the threads may proceed.


\clearemptydoublepage
\subsection {Barrier hint}

For many of the problems in this book I will provide hints
by presenting the variables I used in my solution and
explaining their roles.

\begin{lstlisting}[title={Barrier hint}]{}
n = the number of threads
count = 0
mutex = Semaphore(1)
barrier = Semaphore(0)
\end{lstlisting}

{\tt count} keeps track of how many threads have arrived.
{\tt mutex} provides exclusive access to {\tt count} so that
threads can increment it safely.

{\tt barrier} is locked
(zero or negative) until all threads arrive; then it should
be unlocked (1 or more).


\clearemptydoublepage
\subsection {Barrier non-solution}

First I will present a solution that is not quite right, because
it is useful to examine incorrect solutions and figure out what
is wrong.

\begin{lstlisting}[title={Barrier non-solution}]{}
rendezvous

mutex.wait()
    count = count + 1
mutex.signal()

if count == n: barrier.signal()

barrier.wait()

critical point
\end{lstlisting}

Since {\tt count} is protected by a mutex, it counts the number of
threads that pass.  The first $n-1$ threads wait when they get to the
barrier, which is initially locked.  When the $n$th thread arrives, it
unlocks the barrier.

Puzzle:  What is wrong with this solution?


\clearemptydoublepage
\subsection{Deadlock \#2}

The problem is a deadlock.

An an example, imagine that $n=5$
and that 4 threads are waiting at the barrier.  The value
of the semaphore is the number of threads in queue, negated, 
which is -4.

When the 5th thread signals the barrier, one of the waiting
threads is allowed to proceed, and the semaphore is incremented
to -3.

But then no one signals the semaphore again and none of the
other threads can pass the barrier.
This is a second example of a deadlock.

Puzzle: Does this code always create a deadlock?  Can you find an
execution path through this code that does {\em not} cause a deadlock?

Puzzle: Fix the problem.


\clearemptydoublepage
\subsection{Barrier solution}
\label{barrier}

Finally, here is a working barrier:

\begin{lstlisting}[title={Barrier solution}]{}
rendezvous

mutex.wait()
    count = count + 1
mutex.signal()

if count == n: barrier.signal()

barrier.wait()
barrier.signal()

critical point
\end{lstlisting}

The only change is another {\tt signal} after waiting
at the barrier.  Now as each thread passes, it signals the
semaphore so that the next thread can pass.

This pattern, a {\tt wait} and a {\tt signal} in rapid
succession, occurs often enough that it has a name;
it's called a {\bf turnstile}, because it allows one thread to pass
at a time, and it can be locked to bar all threads.

In its initial state (zero), the turnstile is locked.  The $n$th
thread unlocks it and then all $n$ threads go through.

It might seem dangerous to read the value of {\tt count} outside the
mutex.  In this case it is not a problem, but in general it is
probably not a good idea.  We will clean this up in a few pages, but
in the meantime, you might want to consider these questions: After the
$n$th thread, what state is the turnstile in?  Is there any way the
barrier might be signaled more than once?


\clearemptydoublepage
\subsection {Deadlock \#3}

Since only one thread at a time can pass through the
mutex, and only one thread at a time can pass through
the turnstile, it might seen reasonable to put the
turnstile inside the mutex, like this:

\begin{lstlisting}[title={Bad barrier solution}]{}
rendezvous

mutex.wait()
    count = count + 1
    if count == n: barrier.signal()

    barrier.wait()
    barrier.signal()
mutex.signal()

critical point
\end{lstlisting}

This turns out to be a bad idea because it can cause a
deadlock.

Imagine that the first thread enters the
mutex and then blocks when it reaches the turnstile.
Since the mutex is locked, no other threads can enter,
so the condition, {\tt count==n}, will never be true and
no one will ever unlock the turnstile.

In this case the deadlock is fairly obvious, but it
demonstrates a
common source of deadlocks: blocking on a semaphore while
holding a mutex.


\section {Reusable barrier}
\label{rebar}

Often a set of cooperating threads will perform a series of steps
in a loop and synchronize at a barrier after each step.  For this
application we need a reusable barrier that locks itself after
all the threads have passed through.

Puzzle: Rewrite the barrier solution so that after all the threads
have passed through, the turnstile is locked again.


\clearemptydoublepage
\subsection {Reusable barrier non-solution \#1}

Once again, we will start with a simple attempt at a solution
and gradually improve it:

\begin{lstlisting}[title={Reusable barrier non-solution}]{}
rendezvous

mutex.wait()
    count += 1
mutex.signal()

if count == n: turnstile.signal()       (*\label{problem}*)

turnstile.wait()
turnstile.signal()

critical point

mutex.wait()
    count -= 1
mutex.signal()

if count == 0: turnstile.wait()         (*\label{problem2}*)
\end{lstlisting}

Notice that the code after the turnstile is pretty much the same as
the code before it.  Again, we have to use the mutex to protect access
to the shared variable {\tt count}.
Tragically, though, this code is not quite correct.

Puzzle: What is the problem?


\clearemptydoublepage
\subsection {Reusable barrier problem \#1}

There is a problem spot at Line~\ref{problem} of the previous code.

If the $n-1$th thread is interrupted at this point,
and then the $n$th thread comes through the mutex,
both threads will find that {\tt count==n} and both
threads will signal the turnstile.  In fact, it is even
possible that {\em all} the threads will signal the turnstile.

Similarly, at Line~\ref{problem2} it is possible for multiple
threads to {\tt wait}, which will cause a deadlock.

Puzzle: Fix the problem.

\clearemptydoublepage
\subsection {Reusable barrier non-solution \#2}

This attempt fixes the previous error, but a subtle problem
remains.

\begin{lstlisting}[title={Reusable barrier non-solution}]{}
rendezvous

mutex.wait()
    count += 1
    if count == n: turnstile.signal()
mutex.signal()

turnstile.wait()
turnstile.signal()

critical point

mutex.wait()
    count -= 1
    if count == 0: turnstile.wait()
mutex.signal()
\end{lstlisting}

In both cases the check is inside the mutex so that
a thread cannot be interrupted after changing the counter
and before checking it.

Tragically, this code is {\em still} not correct.
Remember that this barrier will be inside a loop.  So, after
executing the last line, each thread will go back
to the rendezvous.

Puzzle: Identify and fix the problem.


\clearemptydoublepage
\subsection {Reusable barrier hint}

As it is currently written, this code
allows a precocious thread to pass through the second mutex,
then loop around and pass through the first mutex and the
turnstile, effectively getting ahead of the other threads by
a lap.

To solve this problem we can use two turnstiles.

\begin{lstlisting}[title={Reusable barrier hint}]{}
turnstile = Semaphore(0)
turnstile2 = Semaphore(1)
mutex = Semaphore(1)
\end{lstlisting}

Initially the first is locked and the second is open.  When all the
threads arrive at the first, we lock the second and unlock the first.
When all the threads arrive at the second we relock the first,
which makes it safe for the threads to loop around to the beginning,
and then open the second.



\clearemptydoublepage
\subsection {Reusable barrier solution}

\begin{lstlisting}[title={Reusable barrier solution}]{}
# rendezvous

mutex.wait()
    count += 1
    if count == n:
        turnstile2.wait()       # lock the second
        turnstile.signal()      # unlock the first
mutex.signal()

turnstile.wait()                # first turnstile
turnstile.signal()

# critical point

mutex.wait()
    count -= 1
    if count == 0: 
        turnstile.wait()         # lock the first
        turnstile2.signal()      # unlock the second
mutex.signal()

turnstile2.wait()                # second turnstile
turnstile2.signal()
\end{lstlisting}

This solution is sometimes called a {\bf two-phase barrier} because
it forces all the threads to wait twice: once for all the threads
to arrive and again for all the threads to execute the critical
section.

Unfortunately, this solution is typical of most non-trivial
synchronization code: it is difficult to be sure that a solution is
correct.  Often there is a subtle way that a particular path through
the program can cause an error.

To make matters worse, testing an implementation of a solution
is not much help.  The error might occur very rarely
because the particular path that causes it might
require a spectacularly unlucky combination of circumstances.
Such errors are almost
impossible to reproduce and debug by conventional means.

The only alternative is to examine the code carefully and
``prove'' that it is correct.  I put ``prove'' in quotation
marks because I don't mean, necessarily, that you have to write
a formal proof (although there are zealots who encourage
such lunacy).

The kind of proof I have in mind is more informal.  We can take
advantage of the structure of the code, and the idioms we have
developed, to assert, and then demonstrate, a number of
intermediate-level claims about the program.  For example:

\begin{enumerate}

\item Only the $n$th thread can lock or unlock the turnstiles.

\item Before a thread can unlock the first turnstile, it has to close 
the second, and vice versa; therefore it is impossible for one thread
to get ahead of the others by more than one turnstile.

\end{enumerate}

By finding the right kinds of statements to assert and
prove, you can sometimes find a concise way to convince yourself
(or a skeptical colleague) that your code is bulletproof.



\clearemptydoublepage
\subsection {Preloaded turnstile}

One nice thing about a turnstile is that it is a versatile
component you can use in a variety of solutions.  But one
drawback is that it forces threads to go through sequentially,
which may cause more context switching than necessary.

In the reusable barrier solution, we can simplify the solution if the
thread that unlocks the turnstile preloads the turnstile with enough
signals to let the right number of threads through\footnote{Thanks to
Matt Tesch for this solution!}.

The syntax I am using here assumes that {\tt signal} can take a
parameter that specifies the number of signals.  This is a
non-standard feature, but it would be easy to implement with a loop.
The only thing to keep in mind is that the multiple signals are not
atomic; that is, the signaling thread might be interrupted in the
loop.  But in this case that is not a problem.

\begin{lstlisting}[title={Reusable barrier solution}]{}
# rendezvous

mutex.wait()
    count += 1
    if count == n:
        turnstile.signal(n)      # unlock the first
mutex.signal()

turnstile.wait()                 # first turnstile

# critical point

mutex.wait()
    count -= 1
    if count == 0:
        turnstile2.signal(n)     # unlock the second
mutex.signal()

turnstile2.wait()                # second turnstile
\end{lstlisting}

When the $n$th thread arrives, it preloads the first turnstile with
one signal for each thread.  When the $n$th thread passes the
turnstile, it ``takes the last token'' and leaves the turnstile locked
again.

The same thing happens at the second turnstile, which is
unlocked when the last thread goes through the mutex.


\newpage
\subsection {Barrier objects}

It is natural to encapsulate a barrier in an object.  I will
borrow the Python syntax for defining a class:

\begin{lstlisting}[title={Barrier class}]{}
class Barrier:
    def __init__(self, n):
        self.n = n
        self.count = 0
        self.mutex = Semaphore(1)
        self.turnstile = Semaphore(0)
        self.turnstile2 = Semaphore(0)

    def phase1(self):
        self.mutex.wait()
            self.count += 1
            if self.count == self.n:
                self.turnstile.signal(self.n) 
        self.mutex.signal()
        self.turnstile.wait()            

    def phase2(self):
        self.mutex.wait()
            self.count -= 1
            if self.count == 0:
                self.turnstile2.signal(self.n)
        self.mutex.signal()
        self.turnstile2.wait()

    def wait(self):
        self.phase1()
        self.phase2()
\end{lstlisting}

The {\tt \_\_init\_\_} method runs when we create a new
{\tt Barrier} object, and initializes the instance variables.
The parameter {\tt n} is the number of threads that have
to invoke {\tt wait} before the Barrier opens.

The variable {\tt self} refers to the object the method
is operating on.  Since each barrier object has its own
mutex and turnstiles, {\tt self.mutex} refers to the specific
mutex of the current object.

Here is an example that creates a {\tt Barrier}
object and waits on it:

\begin{lstlisting}[title={Barrier interface}]{}
barrier = Barrier(n)        # initialize a new barrier
barrier.wait()              # wait at a barrier
\end{lstlisting}

Optionally, code that uses a barrier can call {\tt phase1} and
{\tt phase2} separately, if there is something else that
should be done in between.



\section{Queue}
\label{dancers}

Semaphores can also be used to represent a queue.  In this
case, the initial value is 0, and usually the code is written
so that it is not possible
to signal unless there is a thread waiting, so the value of the
semaphore is never positive.

For example, imagine that threads represent ballroom dancers
and that two kinds of dancers, leaders and followers, wait
in two queues before entering the dance floor.  When a leader
arrives, it checks to see if there is a follower waiting.
If so, they can both proceed.  Otherwise it waits.

Similarly, when a follower arrives, it checks for a leader and
either proceeds or waits, accordingly.

Puzzle: write code for leaders and followers that enforces these
constraints.



\clearemptydoublepage
\subsection {Queue hint}

Here are the variables I used in my solution:

\begin{lstlisting}[title={Queue hint}]{}
leaderQueue = Semaphore(0)
followerQueue = Semaphore(0)
\end{lstlisting}

{\tt leaderQueue} is the queue where leaders wait 
and {\tt followerQueue} is the queue where followers wait.



\clearemptydoublepage
\subsection {Queue solution}

Here is the code for leaders:

\begin{lstlisting}[title={Queue solution (leaders)}]{}
followerQueue.signal()
leaderQueue.wait()
dance()
\end{lstlisting}

And here is the code for followers:

\begin{lstlisting}[title={Queue solution (followers)}]{}
leaderQueue.signal()
followerQueue.wait()
dance()
\end{lstlisting}

This solution is about as simple as it gets; it is just a Rendezvous.
Each leader signals exactly one follower, and each follower signals
one leader, so it is guaranteed that leaders and followers are
allowed to proceed in pairs.  But whether they actually proceed in
pairs is not clear.  It is possible for any number of threads to
accumulate before executing {\tt dance}, and so it is possible for
any number of leaders to {\tt dance} before any followers do.
Depending on the semantics of {\tt dance}, that behavior may or
may not be problematic.

To make things more interesting, let's add the additional constraint
that each leader can invoke {\tt dance} concurrently with only
one follower, and vice versa.  In other words, you got to dance
with the one that brought you\footnote{Song lyric performed by Shania
Twain}.

Puzzle: write a solution to this ``exclusive queue'' problem.

\clearemptydoublepage
\subsection {Exclusive queue hint}

Here are the variables I used in my solution:

\begin{lstlisting}[title={Queue hint}]{}
leaders = followers = 0
mutex = Semaphore(1)
leaderQueue = Semaphore(0)
followerQueue = Semaphore(0)
rendezvous = Semaphore(0)
\end{lstlisting}

{\tt leaders} and {\tt followers} are counters that
keep track of the number of dancers of each kinds that are
waiting.  The mutex guarantees exclusive access to the counters.

{\tt leaderQueue} and {\tt followerQueue} are the queues where
dancers wait.  {\tt rendezvous} is used to check that both threads
are done dancing.



\clearemptydoublepage
\subsection {Exclusive queue solution}

Here is the code for leaders:

\begin{lstlisting}[title={Queue solution (leaders)}]{}
mutex.wait()
if followers > 0:
    followers--
    followerQueue.signal()
else:
    leaders++
    mutex.signal()
    leaderQueue.wait()    

dance()
rendezvous.wait()
mutex.signal()
\end{lstlisting}

When a leader arrives, it gets the mutex that protects {\tt leaders}
and {\tt followers}.  If there is a follower waiting, the leader
decrements {\tt followers}, signals a follower, and then invokes
{\tt dance}, all before releasing {\tt mutex}.  That guarantees that
there can be only one follower thread running {\tt dance}
concurrently.

If there are no followers waiting, the leader has to give up the mutex
before waiting on {\tt leaderQueue}.

The code for followers is similar:

\begin{lstlisting}[title={Queue solution (followers)}]{}
mutex.wait()
if leaders > 0:
    leaders--
    leaderQueue.signal()
else:
    followers++
    mutex.signal()
    followerQueue.wait()    

dance()
rendezvous.signal()
\end{lstlisting}

When a follower arrives, it checks for a waiting leader.  If there
is one, the follower decrements {\tt leaders}, signals a leader, and
executes {\tt dance}, all without releasing {\tt mutex}.  Actually,
in this case the follower {\em never} releases {\tt mutex};
the leader does.  We don't have to keep track of which thread has the
mutex because we know that one of them does, and either one of them can
release it.  In my solution it's always the leader.

When a semaphore is used as a queue\footnote{A semaphore used as a
queue is very similar to a condition variable.  The primary difference
is that threads have to release the mutex explicitly before waiting,
and reacquire it explicitly afterwards (but only if they need it).},
I find it useful to read ``wait'' as ``wait for this queue'' and
signal as ``let someone from this queue go.''

In this code we never signal a queue unless someone is waiting,
so the values of the queue semaphores are seldom positive.
It is possible, though.  See if you can figure out how.


\clearemptydoublepage
