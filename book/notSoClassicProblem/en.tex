\chapter{Not-so-classical problems}

\section{The search-insert-delete problem}

This one is from Andrews's {\em Concurrent Programming} \cite{andrews}.

\begin {quotation}
Three kinds of threads share access to a singly-linked list:
searchers, inserters and deleters.  Searchers merely examine the list;
hence they can execute concurrently with each other.  Inserters add
new items to the end of the list; insertions must be mutually
exclusive to preclude two inserters from inserting new items at about
the same time.  However, one insert can proceed in parallel with any
number of searches.  Finally, deleters remove items from anywhere in
the list.  At most one deleter process can access the list at a time,
and deletion must also be mutually exclusive with searches and
insertions.
\end{quotation}

Puzzle: write code for searchers, inserters and deleters that
enforces this kind of three-way categorical mutual exclusion.


\clearemptydoublepage
\subsection{Search-Insert-Delete hint}

\begin{lstlisting}[title={Search-Insert-Delete hint}]{}
insertMutex = Semaphore(1)
noSearcher = Semaphore(1)
noInserter = Semaphore(1)
searchSwitch = Lightswitch()
insertSwitch = Lightswitch()
\end{lstlisting}

{\tt insertMutex} ensures that only one inserter is in its critical
section at a time.  {\tt noSearcher} and {\tt noInserter} indicate
(surprise) that there are no searchers and no inserters in their
critical sections; a deleter needs to hold both of these to enter.

{\tt searchSwitch} and {\tt insertSwitch} are used by searchers and
inserters to exclude deleters.


\clearemptydoublepage
\subsection{Search-Insert-Delete solution}

Here is my solution:

\begin{lstlisting}[title={Search-Insert-Delete solution (searcher)}]{}
searchSwitch.wait(noSearcher)
# critical section
searchSwitch.signal(noSearcher)
\end{lstlisting}

The only thing a searcher needs to worry about is a deleter.
The first searcher in takes {\tt noSearcher}; the last one out
releases it.

\begin{lstlisting}[title={Search-Insert-Delete solution (inserter)}]{}
insertSwitch.wait(noInserter)
insertMutex.wait()
# critical section
insertMutex.signal()
insertSwitch.signal(noInserter)
\end{lstlisting}

Similarly, the first inserter takes {\tt noInserter} and the last one
out releases it.  Since searchers and inserters compete for different
semaphores, they can be in their critical section concurrently.
But {\tt insertMutex} ensures that only one inserter is in the room
at a time.

\begin{lstlisting}[title={Search-Insert-Delete solution (deleter)}]{}
noSearcher.wait()
noInserter.wait()
# critical section
noInserter.signal()
noSearcher.signal()

\end{lstlisting}

Since the deleter holds both {\tt noSearcher} and {\tt noInserter},
it is guaranteed exclusive access.  Of course, any time we see
a thread holding more than one semaphore, we need to check for
deadlocks.  By trying out a few scenarios, you should be able
to convince yourself that this solution is deadlock free.

On the other hand, like many categorical exclusion problems, this
one is prone to starvation.  As we saw in the Readers-Writers problem,
we can sometimes mitigate this problem by giving priority to one
category of threads according to application-specific criteria.
But in general it is difficult to write an efficient solution
(one that allows the maximum degree of concurrency)
that avoids starvation.



\section{The unisex bathroom problem}

I wrote this problem\footnote{Later I learned that a nearly
identical problem appears in Andrews's 
{\em Concurrent Programming}\cite{andrews}} when a friend of mine
left her position teaching physics at Colby College
and took a job at Xerox.

She was working in a cubicle in the basement of a
concrete monolith, and the nearest women's bathroom was two floors up.
She proposed to the Uberboss that they convert the men's bathroom
on her floor to a unisex bathroom, sort of like on Ally McBeal.

The Uberboss agreed, provided that the following synchronization
constraints can be maintained:

\begin {itemize}

\item There cannot be men and women in the bathroom
at the same time.

\item There should never be more than three
employees squandering company time in the bathroom.

\end{itemize}

Of course the solution should avoid deadlock.  For now, though, don't
worry about starvation.  You may assume that the bathroom is equipped
with all the semaphores you need.


\clearemptydoublepage
\subsection {Unisex bathroom hint}

Here are the variables I used in my solution:

\begin{lstlisting}[title={Unisex bathroom hint}]{}
empty = Semaphore(1)
maleSwitch = Lightswitch()
femaleSwitch = Lightswitch()
maleMultiplex = Semaphore(3)
femaleMultiplex = Semaphore(3)
\end{lstlisting}

{\tt empty} is 1 if the room is empty and 0 otherwise.

{\tt maleSwitch} allows men to bar women from the room.
When the first male enters, the lightswitch locks {\tt empty}, barring women;
When the last male exits, it unlocks {\tt empty}, allowing women
to enter.  Women do likewise using {\tt femaleSwitch}.

{\tt maleMultiplex} and {\tt femaleMultiplex} ensure that there are no
more than three men and three women in the system at a time.


\clearemptydoublepage
\subsection {Unisex bathroom solution}

Here is the female code:

\begin{lstlisting}[title={Unisex bathroom solution (female)}]{}
femaleSwitch.lock(empty)
    femaleMultiplex.wait()
        # bathroom code here
    femaleMultiplex.signal()
female Switch.unlock(empty)
\end{lstlisting}

The male code is similar.

Are there any problems with this solution?

\clearemptydoublepage
\subsection {No-starve unisex bathroom problem}

The problem with the previous solution is that it allows starvation.
A long line of women can arrive and enter while there is a man
waiting, and vice versa.

Puzzle: fix the problem.


\clearemptydoublepage
\subsection {No-starve unisex bathroom solution}

As we have seen before, we can use a turnstile to allow one
kind of thread to stop the flow of the other kind of thread.
This time we'll look at the male code:

\begin{lstlisting}[title={No-starve unisex bathroom solution (male)}]{}
turnstile.wait()
    maleSwitch.lock(empty)
turnstile.signal()

    maleMultiplex.wait()
        # bathroom code here
    maleMultiplex.signal()

maleSwitch.unlock (empty)
\end{lstlisting}

As long as there are men in the room, new arrivals will pass
through the turnstile and enter.  If there are women in the room
when a male arrives, the male will block inside the turnstile,
which will bar all later arrivals (male and female) from entering
until the current occupants leave.  At that point the male in
the turnstile enters, possibly allowing additional males to enter.

The female code is similar, so if there are men in the room an
arriving female will get stuck in the turnstile, barring additional
men.

This solution may not be efficient.  If
the system is busy, then there will often be several threads, male and
female, queued on the turnstile.  Each time {\tt empty} is signaled,
one thread will leave the turnstile and another will enter.  If the
new thread is the opposite gender, it will promptly block, barring
additional threads.  Thus, there will usually be only 1-2 threads in
the bathroom at a time, and the system will not take full advantage
of the available concurrency.


\section{Baboon crossing problem}

This problem is adapted from Tanenbaum's {\em Operating Systems:
Design and Implementation} \cite{tanenbaum}.
There is a deep canyon somewhere in
Kruger National Park, South Africa, and a single rope that spans the
canyon.  Baboons can cross the canyon by swinging hand-over-hand on
the rope, but if two baboons going in opposite directions meet in the
middle, they will fight and drop to their deaths.  Furthermore,
the rope is only strong enough to hold 5 baboons.  If there are
more baboons on the rope at the same time, it will break.

Assuming that we can teach the baboons to use semaphores, we
would like to design a synchronization scheme with the following
properties:

\begin{itemize}

\item Once a baboon has begun to cross, it is guaranteed
to get to the other side without running into a baboon going
the other way.

\item There are never more than 5 baboons on the rope.

\item A continuing stream of baboons crossing in one direction
should not bar baboons going the other way indefinitely
(no starvation).

\end{itemize}

I will not include a solution to this problem for reasons that
should be clear.


\section{The Modus Hall Problem}

This problem was written by Nathan Karst, one of the Olin students
living in Modus Hall\footnote{Modus Hall is one of several nicknames
for the modular buildings, aka Mods, that some students lived in while
the second residence hall was being built.} during the winter of 2005.

\begin{quote}
After a particularly heavy snowfall this winter, the denizens of Modus
Hall created a trench-like path between their cardboard shantytown and
the rest of campus.  Every day some of the residents walk to and from
class, food and civilization via the path; we will ignore the
indolent students who chose daily to drive to Tier 3.  We will also
ignore the direction in which pedestrians are traveling.  For some
unknown reason, students living in West Hall would occasionally find it
necessary to venture to the Mods.

Unfortunately, the path is not wide enough to allow two people
to walk side-by-side.  If two Mods persons meet at some point on the
path, one will gladly step aside into the neck high drift to accommodate
the other.  A similar situation will occur if two ResHall inhabitants
cross paths.  If a Mods heathen and a ResHall prude meet, however, a
violent skirmish will ensue with the victors determined solely by
strength of numbers; that is, the faction with the larger population will
force the other to wait.
\end{quote}

This is similar to the Baboon Crossing problem (in more ways than
one), with the added twist that control of the critical section is
determined by majority rule.  This has the potential to be an
efficient and starvation-free solution to the categorical exclusion
problem.

Starvation is avoided because while one faction controls the critical
section, members of the other faction accumulate in queue until they
achieve a majority.  Then they can bar new opponents from entering
while they wait for the critical section to clear.  I expect this
solution to be efficient because it will tend to move threads through
in batches, allowing maximum concurrency in the critical section.

Puzzle: write code that implements categorical exclusion with
majority rule.



\clearemptydoublepage
\subsection {Modus Hall problem hint}

Here are the variables I used in my solution.

\begin{lstlisting}[title={Modus problem hint}]{}
heathens = 0
prudes = 0
status = 'neutral'
mutex = Semaphore(1)
heathenTurn = Semaphore(1)
prudeTurn = Semaphore(1)
heathenQueue = Semaphore(0)
prudeQueue = Semaphore(0)
\end{lstlisting}

{\tt heathens} and {\tt prudes} are counters, and {\tt status} records
the status of the field, which can be `neutral', `heathens rule',
`prudes rule', `transition to heathens' or `transition to prudes'.
All three are protected by {\tt mutex} in the usual scoreboard
pattern.

{\tt heathenTurn} and {\tt prudeTurn} control access to the field
so that we can bar one side or the other during a transition.

{\tt heathenQueue} and {\tt prudeQueue} are where threads wait after
checking in and before taking the field.


\clearemptydoublepage
\subsection {Modus Hall problem solution}

Here is the code for heathens:

\begin{lstlisting}[title={Modus problem solution}]{}
heathenTurn.wait()
heathenTurn.signal()
(*\label{silentmajority}*)
mutex.wait()
heathens++

if status == 'neutral':
    status = 'heathens rule'
    mutex.signal()
elif status == 'prudes rule':
    if heathens > prudes:
        status = 'transition to heathens'
        prudeTurn.wait()
    mutex.signal()
    heathenQueue.wait()
elif status == 'transition to heathens':
    mutex.signal()
    heathenQueue.wait()
else
    mutex.signal()

# cross the field

mutex.wait()
heathens--

if heathens == 0:
    if status == 'transition to prudes':
        prudeTurn.signal()
    if prudes:
        prudeQueue.signal(prudes)
        status = 'prudes rule'
    else:
        status = 'neutral'
        
if status == 'heathens rule':
    if prudes > heathens:
        status = 'transition to prudes'
        heathenTurn.wait()

mutex.signal()
\end{lstlisting}

As each student checks in, he has to
consider the following cases:

\begin{itemize}

\item If the field is empty, the student lays claim for the heathens.

\item If the heathens currently in charge, but the new arrival
has tipped the balance, he locks the prude turnstile and the
system switches to transition mode.

\item If the prudes in charge, but the new arrival doesn't
tip the balance, he joins the queue.

\item If the system is transitioning to heathen control, the new arrival
joins the queue.

\item Otherwise we conclude that either the heathens are in charge, or the
system is transitioning to prude control.  In either case, this
thread can proceed.

\end{itemize}  

Similarly, as each student checks out, she has to consider several
cases.  

\begin{itemize}

\item If she is the last heathen to check out, she has to
consider the following:

    \begin{itemize}

    \item If the system is in transition, that means that the prude
      turnstile is locked, so she has to open it.

    \item If there are prudes waiting, she signals them and
      updates {\tt status} so the prudes are in charge.  If not, the
      new status is 'neutral'.

    \end{itemize}  

\item If she is not the last heathen to check out, she still has to
check the possibility that her departure will tip the balance.  In
that case, she closes the heathen turnstile and starts the
transition.

\end{itemize}

One potential difficulty of this solution is that any number
of threads could be interrupted at Line~\ref{silentmajority},
where they would have passed the turnstile but not yet checked in.
Until they check in, they are not counted, so the balance of
power may not reflect the number of threads that have passed the
turnstile.  Also, a transition ends when all the threads that have
checked in have also checked out.  At that point, there may
be threads (of both types) that have passed the turnstile.

These behaviors may affect efficiency---this solution does
not guarantee maximum concurrency---but they don't affect
correctness, if you accept that ``majority rule'' only applies
to threads that have registered to vote.


\clearemptydoublepage
