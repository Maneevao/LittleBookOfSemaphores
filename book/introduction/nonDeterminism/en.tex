\section {Non-determinism}

Concurrent programs are often {\bf non-deterministic}, which means it
is not possible to tell, by looking at the program, what will happen
when it executes.  Here is a simple example of a
non-deterministic program:

\begin{minipage}[t]{2in}
\begin{lstlisting}[title={Thread A}]{}
print "yes"
\end{lstlisting}
\end{minipage}
\hfill
\begin{minipage}[t]{2in}
\begin{lstlisting}[title={Thread B}]{}
print "no"
\end{lstlisting}
\end{minipage}

Because the two threads run concurrently, the order of
execution depends on the scheduler.  During any given run
of this program, the output might be ``yes no'' or ``no yes''.

Non-determinism is one of the things that makes concurrent
programs hard to debug.  A program might work correctly
1000 times in a row, and then crash on the 1001st run, depending
on the particular decisions of the scheduler.

These kinds of bugs are almost impossible to find by testing;
they can only be avoided by careful programming.
