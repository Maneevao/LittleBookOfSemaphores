\section{Synchronization}
\label{synch}

In common use, ``synchronization'' means making two things happen
at the same time.  In computer systems, synchronization is a little
more general; it refers to relationships among events---any number
of events, and any kind of relationship (before, during, after).

Computer programmers are often concerned with {\bf synchronization
constraints}, which are requirements pertaining to the order of
events.  Examples include:

\begin{description}

\item[Serialization:] Event A must happen before Event B.

\item[Mutual exclusion:] Events A and B must not happen at the same time.

\end{description}

In real life we often check and enforce synchronization constraints 
using a clock.  How do we know if A happened before B?  If we
know what time both events occurred, we can just compare the times.

In computer systems, we often need to satisfy synchronization
constraints without the benefit of a clock, either because there
is no universal clock, or because we don't know with fine enough
resolution when events occur.

That's what this book is about: software techniques for enforcing
synchronization constraints.
