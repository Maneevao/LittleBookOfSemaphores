\section{Синхронизация}
\label{synch}

Обычно ``синхронизация'' обозначает выполнение двух событий
в одно и то же время.
В компьютерных системах синхронизация является более обобщённым понятием,
которое говорит об отношении между событиями --- 
любом количестве событий, любом типе отношения (до, в течении, после).

Программисты часто обеспокоены о {\bf границах синхронизации},
требования которых касаются последовательности событий.
Например:

\begin{description}

\item[Серия:] Событие А должно случиться до события Б.

\item[Взаимное исключение:] Событие А не должно происходить одновременно с
событием Б.

\end{description}

В реальной жизни мы часто проверяем и уточняем границы синхронизации,
используя часы.
Как мы можем знать, что А случилось до Б?
Если мы знаем время, когда случились события, мы можем просто сравнить его.

В компьютерных системах нам часто требуется выполнять синхронизацию
без помощи часов, так как или нет универсальных часов,
или мы не знаем с достаточной точностью, когда события произошли.

Это то, о чём эта книга: программные техники, позволяющие выполнить
синхронизацию.
