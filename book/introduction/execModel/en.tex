\section {Execution model}

In order to understand software synchronization, you have to
have a model of how computer programs run.  In the simplest
model, computers execute one instruction after another in
sequence.  In this model, synchronization is trivial; we can
tell the order of events by looking at the program.  If Statement
A comes before Statement B, it will be executed first.

There are two ways things get more complicated.  One possibility
is that the computer is parallel, meaning that it has multiple
processors running at the same time.  In that case it is not easy
to know if a statement on one processor is executed before a
statement on another.

Another possibility is that a single processor is running multiple
threads of execution.  A thread is a sequence of instructions
that execute sequentially.  If there are multiple threads, then
the processor can work on one for a while, then switch to
another, and so on.

In general the programmer has no control over when each thread runs;
the operating system (specifically, the scheduler) makes those
decisions.  As a result, again, the programmer can't tell when
statements in different threads will be executed.

For purposes of synchronization, there is no difference between the
parallel model and the multithread model.  The issue is the
same---within one processor (or one thread) we know the order of
execution, but between processors (or threads) it is impossible to
tell.

A real world example might make this clearer.  Imagine that you and
your friend Bob live in different cities, and one day, around dinner
time, you start to wonder who ate lunch first that day, you or Bob.
How would you find out?

Obviously you could call him and ask what time he ate lunch.  But what
if you started lunch at 11:59 by your clock and Bob started lunch at
12:01 by his clock?  Can you be sure who started first?  Unless you
are both very careful to keep accurate clocks, you can't.

Computer systems face the same problem because, even though their
clocks are usually accurate, there is always a limit to their
precision.  In addition, most of the time the computer does not keep
track of what time things happen.  There are just too many things
happening, too fast, to record the exact time of everything.

Puzzle: Assuming that Bob is willing to follow simple instructions, is
there any way you can {\em guarantee} that tomorrow you will eat lunch
before Bob?

\clearemptydoublepage
