\section {Execution model}

Для понимания программной синхронизации, вы должны понимать,
как работают компьютерные программы.
В простейшей модели компьютер исполняет одну инструкцию
за другой в последовательности.
В этой модели синхронизация тривиальна, мы можем сказать
последовательность событий, просто посмотрев на программу.
Если событие А происходит до события Б, одно исполнится первым.

Есть два способа усложнить задачу.
Первый связан с  параллельными вычислительными системы, где
несколько процессоров работают в одно и то же время.
В этом случае не легко подтвердить, что событие на одном процессоре
исполняется до исполнения события на другом.

Другая возможность связана с тем, что на одном процессоре может
быть запущено несколько потоков исполнения.
Поток --- это последовательность инструкций, которая исполняется последовательно.
Если есть несколько потоков, тогда процессор может выполнять один поток в 
течении некоторого времени, а потом перейти на другой и так далее.

В общем случае, программист не может контролировать,
когда какой поток исполнять.
Операционная система (в частности, планировщик задач) занимается этим вопросом.
В итоге, программист не может сказать,
когда события в разных потоках будут выполнены.

Для целей задачи синхронизации нет разницы между параллельными и многопоточными
моделями исполнения программ.
Исходные данные схожи --- в рамках одного процессора (или одного потока) мы 
знаем последовательность исполнения, но между процессорами (или потоками) её
невозможно определить.

Пример из реальной жизни поможет упростить понимание.
Представьте, что вы и ваш друг Боб живёте в разных городах, и однажды, перед
ужином мы начали волноваться, кто съел ланч первым, вы или Боб.

Как вы могли бы узнать?

Очевидно, вы могли бы позвонить ему и спросить, когда он съел ланч.
But what
Но что, если вы начали ланч в 11:59 по вашим часам, а Боб в 12:01 по его часам?
Можете ли вы быть уверены в том, кто начал первым?
Вы не можете,
если вы оба недостаточно аккуратны, чтобы поддерживать точные часы.

Компьютерные системы сталкиваются со схожими проблемами, обычно,
даже с точными часами. Всегда есть лимит по их точности.
В дополнение, большую часть времени компьютер не следит за временем, когда
события происходят.
Слишком много событий происходят слишком быстро, чтобы регистрировать точное
время для всего.

Задача: Предположим, что Боб готов выполнить простые инструкции, есть ли
способ гарантирующий то, что завтра вы будете есть ланч перед Бобом?

\clearemptydoublepage
