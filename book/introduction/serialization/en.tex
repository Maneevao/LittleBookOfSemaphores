\section {Serialization with messages}
\label{serialization}

One solution is to instruct Bob not to eat lunch until you call.
Then, make sure you don't call until after lunch.  This approach may
seem trivial, but the underlying idea, message passing, is a real
solution for many synchronization problems.
At the risk of belaboring the obvious, consider this timeline.
%
\begin{minipage}[t]{2in}
\begin{lstlisting}[title={Thread A (You)}]{}
Eat breakfast 
Work          
Eat lunch     
Call Bob
\end{lstlisting}
\end{minipage}
\hfill
\begin{minipage}[t]{2in}
\begin{lstlisting}[title={Thread B (Bob)}]{}
Eat breakfast
Wait for a call
Eat lunch
\end{lstlisting}
\end{minipage}
%
The first column is a list of actions you perform; in other words,
your thread of execution.  The second column is Bob's thread of
execution.  Within a thread, we can always tell what order things
happen.  We can denote the order of events
%
\begin{eqnarray*}
a1 < a2 < a3 < a4  \\
b1 < b2 < b3
\end{eqnarray*}
%
where the relation $a1 < a2$ means that a1 happened before a2.

In general, though, there is no way to compare events from different
threads; for example, we have no idea who ate breakfast first (is $a1
< b1$?).

But with message passing (the phone call) we {\em can} tell who ate
lunch first ($a3 < b3$).  Assuming that Bob has no other friends, he
won't get a call until you call, so $b2 > a4$ .  Combining all the
relations, we get
%
\begin{eqnarray*}
b3 > b2 > a4 > a3
\end{eqnarray*}
%
which proves that you had lunch before Bob.

In this case, we would say that you and Bob ate lunch
{\bf sequentially}, because we know the order of events, and you
ate breakfast {\bf concurrently}, because we don't.

When we talk about concurrent events, it is tempting to say
that they happen at the same time, or simultaneously.  As a
shorthand, that's fine, as long as you remember the strict
definition:

\begin{quote}
Two events are concurrent if we cannot tell by looking at
the program which will happen first.
\end{quote}

Sometimes we can tell, after the program runs, which happened first,
but often not, and even if we can, there is no guarantee that we will
get the same result the next time.


\newpage

